\documentclass[a4paper,12pt]{article}
\usepackage[T1]{fontenc}
\usepackage[latin9]{inputenc}
\usepackage{listings}
\usepackage{amsmath}
\usepackage{mathtools}
\usepackage{hyperref}
\usepackage{mathpazo}
\usepackage{wasysym}
\usepackage{graphicx}
\usepackage{tikz,pgfplots}
\usepackage[colorinlistoftodos]{todonotes}
\usepackage{natbib} 
\usepackage{geometry}
\usetikzlibrary{fit,shapes.misc,snakes}
\geometry{verbose,tmargin=2.5cm,bmargin=2.5cm,lmargin=2.5cm,rmargin=2.5cm}

\newcommand{\ord}{\operatorname{ord}}

\graphicspath{ {../Project1/measurements/} }

\begin{document}

\title{I/O-algorithms\\Project 1}

\author{Lasse Espeholt - 20093223\\
Kasper Nielsen - 20091182\\}

\maketitle
\begin{figure}[h!]
\includegraphics[width=\textwidth]{"images/forside"}
\end{figure}


\vfill{}
\begin{description}
\item [{Implementation~code~and~test~results:}]~
\\
\texttt{\url{https://github.com/kasper0406/IO13/tree/master/Project1/}}
\end{description}
\pagebreak{}\tableofcontents{}\pagebreak{}

\section{Introduction}
In this report we will discuss how we implemented and benchmarked the external heap and the different underlysing streams. Different parameters and algorithm variations will be examined in order to optimize the running time of sorting and for comparison with in-memory sorting algorithms and the external mergesort.

\section{Setup}
This section presents the test setup, how measurements were performed
and gives an overview of the files attached to this report.

\subsection{Test setup}
%!TEX root = rapport.tex
All measurements presented in this report were performed on a computer
with an Intel Core i3-380UM CPU with the following technical
specifications:
\begin{itemize}
\item 2 cores w/HT operating at 1.33 GHz
\item 2 x 32 kB L1 data cache
\item 2 x 256 kB L2 cache
\item Shared 3 MB L3 cache
\end{itemize}
The main memory size of the machine was 1 GB (with approximately 600 MB free memory), and the machine was running Ubuntu
Linux 12.04. The hard drive was an USB2 attached Fujitsu MJA2250BH which had the following specifications:

\begin{itemize}
\item 250 GB
\item 5,400 RPM
\item 8 MB cache
\item 512 B sector size
\end{itemize}

The file system had a block size of 4096 B and 128 kB read ahead.

All implementations have been
written in C++11 and compiled on Linux using the GCC 4.7 compiler. The
code was compiled using the \texttt{-O3 -flto -funroll-loops}
optimization flags.

All tests have been performed with uniform distributed random generated 32 bit integers. The variations between several runs were insignificant so we choose to run all tests once in order to run as many configurations as possible.

We measured I/Os with \texttt{/proc/diskstats} and before each test we ran \texttt{sync()}, forcing a write of dirty buffers in the block buffer cache out to disk.

\subsection{File structure}
The following is a description of the different folders available at
\\
\texttt{\url{https://github.com/kasper0406/IO13/tree/master/Project1/}}
\begin{description}
\item[root] Contains the implementations of the different streams and
  the external mergesort.

\item[measurements/] Contains the raw measurements used for the plots in the
  report. The plots themselves are also included in a pdf-format.
\end{description}

\subsubsection{Code structure}
The code has the following source files:
\begin{description}
\item[main.cpp] Driver code for calling test code.
\item[test.h] Test framework for testing streams and sorting with
  different parameters.
\item[utils.h] Responsible for generating large files for testing.
\item[CMakeLists.txt] CMake file for the project specifying
  compilation options.

\item[rw\_stream.h] Shared code for streams using \texttt{write} and
  \texttt{read} system calls. The unbuffered streams are implemented
  in:
  \begin{itemize}
  \item read\_input\_stream.h
  \item write\_output\_stream.h
  \end{itemize}
  The buffered streams are implemented in:
  \begin{itemize}
  \item buffered\_input\_stream.h
  \item buffered\_output\_stream.h
  \end{itemize}

\item[f\_stream.h] Shared code for streams using \texttt{fwrite} and
  \texttt{fread} functions from the C standard library. Specific
  stream implementations are:
  \begin{itemize}
  \item fread\_input\_stream.h
  \item fwrite\_output\_stream.h
  \end{itemize}

\item[mmap\_stream.h] Shared code for streams using memory mapping of
  files. Specific stream implementations are:
  \begin{itemize}
  \item mmap\_input\_stream.h
  \item mmap\_output\_stream.h
  \end{itemize}
\end{description}

\section{Streams}
This section presents how benchmarks of the four different
streams were performed, and also contains an analysis of their performance. First each stream will
be treated separately and the best settings for each stream will be
used in a relative performance benchmark for finding the stream most
suitable for sorting.

In order to mimic the behaviour of sorting during the test, we
tested the streams in respectively reading/writing $N$ elements from/to a file. Each test uses $k$ interleaved streams, such
that a new stream is starting at every $\frac{N}{k}$ elements, and
elements are being processed in a round robin fashion.

All tests of the streams have used $N = 2^{28}$, which corresponds to
1GB of elements. This choice was made to make sure the data could not be hold in memory in its entirety, and still small enough to be able to run all the tests within reasonably time. The
number of elements has not been varied for the stream test, as it is not
expected that streams will behave differently, as long as the data is
big enough to not fit the main memory.

In the merge step of external mergesort, the input streams are
used similarly as in the experiments. However, the output streams are only used
as in the $k = 1$ case, but we have chosen to do testing of these streams
for varying values of $k$, to be able to get comparable results
between input and output streams.

\subsection{Buffered streams}
\label{sec:buffered-streams}
This type of streams are implemented using the \texttt{read} and
\texttt{write} POSIX system calls. On top of those, a buffer of size
$B$ is added such that the system calls are only called once for $B$ elements. This section investigates how the choice of buffer size and varying
values of $k$ influences the
performance.
Figure~\ref{fig:buffered-input} shows a plot of the performance for the buffered input stream.

\begin{figure}[h!]
  \centering
  \includegraphics[width=0.8\textwidth]{buffered_input}
  \caption{Running times for the buffered input stream using different
    buffer sizes and varying $k$-values.}
  \label{fig:buffered-input}
\end{figure}

From Figure~\ref{fig:buffered-input} it seems like all the streams
perform similarly for sequential read ($k = 1$). This is most likely because the operation system and the disk controller easily can classify the access pattern as sequential and hence, can cache the subsequent blocks with very low cost.

As $k$ gets bigger, Figure~\ref{fig:buffered-input} indicates that
bigger buffers gives better performance, which was expected. However,
for large buffer sizes in combination with large values of $k$, the
input stream performs extremely poorly. The decrease in performance happens because the
buffers partly ends up on disk because of insufficient memory which causes memory thrashing. Experiments with larger buffer sizes were also performed and they
showed a similar behaviour and are therefore of no
interest for use in sorting.


Figure~\ref{fig:buffered-input} shows a small curve for smaller buffer
sizes. As $k$ increases, more disk seeking is required, but when $k$
becomes large, the data of the $k$ different streams is close to each other,
especially because of 128kB read ahead per stream.

An experiment was performed to check this hypothesis, where we reduced the amount of read ahead to 64kB. Figure~\ref{fig:buffered-readahead} shows a plot of the
results. It is seen that the curve is skewed compared to before. This
provides evidence that read ahead has an effect when $k$ is large. For
smaller $k$ values the performance seems almost the same, which we
speculate is due to the look-ahead done by the disk controller. This
only works for small $k$, because of a limited cache on the disk
controller.

\begin{figure}
  \centering
  \includegraphics[width=0.8\textwidth]{buffered_input_rd}
  \caption{Effect on changing the read ahead value for buffered input
    stream with a 2MB buffer.}
  \label{fig:buffered-readahead}
\end{figure}

From Figure~\ref{fig:buffered-input}, it seems that a buffer size of 2
MB provides the best compromise of running time and memory usage for
the input stream in the sorting application.

For the output stream a similar plot is shown in
Figure~\ref{fig:buffered-output}.

\begin{figure}[h!]
  \centering
  \includegraphics[width=0.8\textwidth]{buffered_output}
  \caption{Running times for the buffered output stream using different
    buffer sizes and varying $k$-values.}
  \label{fig:buffered-output}
\end{figure}

We expected that the plots in Figure~\ref{fig:buffered-input} and
Figure~\ref{fig:buffered-output} would be similar since the two streams
are much alike. However, the plots shows that the varying the buffer sizes
almost have no effect on the performance of the output streams, except
when they are too big to be in memory. Comparing with buffered input streams we see the large decrease in performance earlier with buffered output streams. We believe it is due to a large write buffer in the operating system to facilitate asynchronous write. Therefore, less memory is available for the output streams.
Furthermore, the disk controller has the choice of delaying writes to the disk platters by using its 8MB cache and wait until it receives more data in the same region of the disk. It thereby avoids random seeks. This could explain why varying buffer sizes performs equally.

An experiment was performed, where the
\texttt{fsync}\footnote{The \texttt{fsync} system call ensures
  that the file is written to disk immediately and also flushes the cache in the disk controller.} system call was
called immediately after each write operation. It was found that both
the input and output streams performed very similar, hence the absence
of effect in varying buffer sizes showed in
Figure~\ref{fig:buffered-output} can be explained by the write cache.
\\
\\
A buffer size of 2 MB has been chosen to give the best trade-off
between running time and memory consumption.

\subsection{MMap streams}
For the memory mapped streams, the \texttt{mmap} and \texttt{munmap}
system calls are used for mapping a file into memory. Then the file
can be accessed as it was in memory, but the actual loading
is lazy, such that it only happens when it is needed.
Therefore it is expected that the block size mapped into memory each
time \texttt{mmap} is called will not have any significant impact on
the results, since only some additional bookkeeping and remapping is required when using different block sizes.

\begin{figure}[h!]
  \centering
  \includegraphics[width=0.8\textwidth]{mmap_input}
  \caption{Running times for the mmap input stream, varying sizes of
    mapped blocks and $k$-values.}
  \label{fig:mmap-input}
\end{figure}

Figure~\ref{fig:mmap-input} shows a plot of the running times for the
memory mapped input stream. Except for $k = 256$, the plot shows
that the running times for the different block sizes are very close,
which is expected. For the $k = 256$ case, it seems like some choices
of block sizes are better than others, but there does not seem to be an
obvious correlation on how the block sizes influences the running time
in this case.

\begin{figure}[h!]
  \centering
  \includegraphics[width=0.8\textwidth]{mmap_output}
  \caption{Running times for the mmap output stream, varying sizes of
    mapped blocks and $k$-values.}
  \label{fig:mmap-output}
\end{figure}

A similar plot for the mmap output stream is shown at
Figure~\ref{fig:mmap-output}. For the output stream it seems like the
different streams behaves alike for different buffer sizes. Like the
input stream, the performance of the output stream is decreasing for
increasing $k$. However, for $k=256$ we see a decrease in performance which we do not fully understand, but we believe it is due to the synchronization algorithm underlying memory mapped files.

\subsection{Other streams}
\label{sec:other-streams}
This section will discuss the two remaining stream
implementations. The first of these are the streams using the
\texttt{fwrite} and \texttt{fread} C standard library calls, which
themselves do buffering. At our test machine the default buffer size
of \texttt{fread} and \texttt{fwrite} was 8
kB. Figure~\ref{fig:fstreams} shows a plot of their performance.

\begin{figure}[h!]
  \centering
  \includegraphics[width=0.8\textwidth]{fstreams}
  \caption{Running times for the streams using the \texttt{fwrite} and
    \texttt{fread} standard library functions, varying the values of
    $k$.}
  \label{fig:fstreams}
\end{figure}

Comparing Figure~\ref{fig:fstreams} to our buffered streams, they show
a very similar behaviour. This is as expected since the streams
should function in exactly the same way.

We will now turn our attention to the streams using the \texttt{read}
and \texttt{write} system calls directly. Due to extremely poor
performance, only one measurement for the output stream was
obtained. The results are plotted in Figure~\ref{fig:syscall-streams}.

\begin{figure}[h!]
  \centering
  \includegraphics[width=0.8\textwidth]{syscall_streams}
  \caption{Running times for the streams using the \texttt{write} and
    \texttt{read} system calls directly, varying the values of $k$.}
  \label{fig:syscall-streams}
\end{figure}

The input stream has an almost constant running time from $k$ between
1 and 16. However, for $k = 64$ a small increase in running time is
seen, which we believe to be due to cache constraints for look-ahead.

As mentioned earlier, only one measurement was done for the output
stream. It is a bit surprising that the output stream is so much
slower than the input stream. The smallest unit able to be written to
the disk is a sector, which on the test machine is 512B.  We have made experiments that indicates writing to disk is delayed until at least 512B have been
written. For reading we have read ahead which is larger than 512B (128kB), which can explain the
low writing performance.

\subsection{Relative comparison}
In order to determine which stream to use, it is very important that
the input streams performs well for varying values of $k$. This
however, is not as important for the output stream since it is not
interleaved when sorting, but other disk access may happen in between writes to the
output stream, meaning that it is also is preferable if
the output stream performs decently for different $k$ values, as this
indicates that the output stream is more robust for interleaved disk
operations.

In order to chose which stream to use in sorting, we have picked out
the input and output streams from the previous sections, which had the
best performance across all values of $k$. In this section we will
compare them, and single out a winner for use in sorting.

Figure~\ref{fig:best-input} shows the running times for what was found
to be the best input streams. This plot suggests that the buffered
input stream with 2MB buffer is the best choice, whereby we have
chosen to use this for sorting.

\begin{figure}[h!]
  \centering
  \includegraphics[width=0.8\textwidth]{best_input}
  \caption{Comparison of the best input streams from each of the four
    types of streams.}
  \label{fig:best-input}
\end{figure}

Figure~\ref{fig:best-output} shows a plot of the best output
streams. The only real competition is between the \texttt{fwrite} and
the buffered output stream with 2MB buffer. As discussed in
Section~\ref{sec:other-streams} the streams are functioning in almost
the same way, except they use a different buffer size.

We choose the 2MB buffered output stream since we also choose the corresponding buffered input stream. As indicated by Figure~\ref{fig:best-output} it might have been a better choice to pick a smaller buffer size. Due to time constraints we did not experiment with a smaller buffer size for the output stream.

\begin{figure}[h!]
  \centering
  \includegraphics[width=0.8\textwidth]{best_output}
  \caption{Comparison of the best output streams from each of the four
    types of streams.}
  \label{fig:best-output}
\end{figure}

\section{Sorting}
This section will investigate how the external memory mergesort
performs for different choices of parameters. Given the input size
$N$, the algorithm has two parameters to tweak: The memory size $M$,
and the number of input streams to merge simultaneously ($k$).

There are some constraints on these parameters. It does not make sense
to have $M$ bigger than the actual memory size, since this corresponds
to a conventional in-memory sorting. Similarly it must be that $k \leq
\left\lceil\frac{N}{M}\right\rceil$, since there should be at least $k$ streams to merge.

Figure~\ref{fig:sorting} plots the running time of sorting 1GB of
integers, using different settings for the $M$ and $k$ values. It is
seen, that the measurements seems to form around the same levels of
running times when varying the parameters. This is due to the number
of times the file is read on the disk.

For every level in the merge tree, the file is read and written once
from/to the disk. The height of the merge tree for give parameters is
$h = \log_k \frac{N}{M}$, since every merge merges $k$ streams, and
there are $\frac{N}{M}$ streams initially. The levels seen in
Figure~\ref{fig:sorting} follows this bound very nicely.

There are small variations for the running time of each level, but
these seems almost negligible in the overall picture. This also
suggests that the assumption of the I/O model, where only disk
accesses are counted, indeed seems to be a good model in
practice when the input does not fit into main memory.

For a given choice of parameters, the amount of memory used to sort
the file is given by $M + B\frac{N}{M}$. Hence, the different
measurements with the same height, use a different amount of
memory. This gives another criteria to optimize when choosing
parameters: First the height of the merge tree should be minimized,
then the amount of used memory. There will be some cases where the
minimum height merge tree is not the best thing to choose, since the
parameters obtaining this may not fit the memory of the machine. This
is seen for $M = 2$MB and $k \geq 256$ in Figure~\ref{fig:sorting}.

When sorting 1GB of integers, we have found parameters $M = 32\textrm{MB}, k =
32$ and $M = 64\textrm{MB}, k = 64$ to be the ones using less memory (96MB) and also minimizing the height of the merge tree.

\begin{figure}[h!]
  \centering
  \includegraphics[width=0.8\textwidth]{sorting2}
  \caption{Time for sorting a 1GB file varying the memory size $M$ and
    streams to merge in parallel $k$.}
  \label{fig:sorting}
\end{figure}

\subsection{Comparison with conventional sorting algorithms}
In order to determine how well the external mergesort is performing,
it has been benchmarked against conventional quicksort and heapsort
implementations. The quicksort is from the C standard library, and
the heapsort is from STL. A plot of the running time for the
different sorting algorithms is shown in Figure~\ref{fig:best-sort}. Notice that benchmarks of heapsort after $2^28$ elements and quicksort after $2^29$ elements was halted due to long execution time.

\begin{figure}[h!]
  \centering
  \includegraphics[width=0.8\textwidth]{best_sort}
  \caption{Comparison of the different sorting algorithms varying the input size.}
  \label{fig:best-sort}
\end{figure}

It can be seen that both quicksort and heapsort outperforms the
external mergesort when $n \leq 2^{27}$, which is 512MB of
elements. This is no surprise, since both quick and heapsort are
in-memory sorting algorithms. The test machine was equipped with 1GB memory, hence
the two sorting algorithms can do everything without going to disk,
while the external mergesort by construction have to do merging on
the disk.

The other thing to notice, is that as soon as the input does not fit
the in memory anymore ($n \geq 2^{28}$, 1GB of elements), the external
mergesort beats the quicksort and heapsort hands down. This was also
what we expected, since the external mergesort should use the disk
much more efficiently.

\section{Conclusion}
In the course of the project, four different stream implementations
have been made and tested with an eye to the application of external
merge sort. It was possible to single out a winner among the
implemented streams, which was found it to be buffered streams with
2MB of buffer size.

It was found that the implemented external Mergesort tremendously improved the
performance of sorting large input, compared to conventional main
memory sorting algorithms -- for sorting 1GB of integer elements, we
found a performance increase of a factor\todo{insert impressive
  factor. + ~algoritme + ved 2GB var heap/qsort absurd langsomt}.

The overall findings of this project was therefore in agreement with
what should be expected from the I/O-model for analyzing algorithms.
\todo{meget vigtigt med hoejden af traeet (i.e. antallet af I/O)}

\clearpage{}\bibliographystyle{plain}
\addcontentsline{toc}{section}{\refname}\bibliography{ref}

\end{document}
